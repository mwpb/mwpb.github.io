\documentclass[a4paper,10pt]{article}
\usepackage{fontspec}

\usepackage{polyglossia}

%A Few Useful Packages
\usepackage{marvosym}
\usepackage[bottom=0.5in]{geometry}
\usepackage{fontspec} 					%for loading fonts
\usepackage{xunicode,xltxtra,url,parskip} 	%other packages for formatting
\RequirePackage{color,graphicx}
\usepackage[usenames,dvipsnames]{xcolor}
\usepackage[big]{layaureo} 				%better formatting of the A4 page
% an alternative to Layaureo can be ** \usepackage{fullpage} **
\usepackage{supertabular} 				%for Grades
\usepackage{titlesec}				%custom \section
%\usepackage{microtype}
\usepackage{tabu}
%Setup hyperref package, and colours for links
\usepackage{hyperref}
\definecolor{linkcolour}{rgb}{0,0.2,0.6}
\hypersetup{colorlinks,breaklinks,urlcolor=linkcolour, linkcolor=linkcolour}

%FONTS
\defaultfontfeatures{Mapping=tex-text}
%\setmainfont[SmallCapsFont = Fontin SmallCaps]{Fontin}
%%% modified for Karol Kozioł for ShareLaTeX use
% \setmainfont[
% SmallCapsFont = Fontin-SmallCaps.otf,
% BoldFont = Fontin-Bold.otf,
% ItalicFont = Fontin-Italic.otf
% ]
% {Fontin.otf}
%%%

%CV Sections inspired by: 
%http://stefano.italians.nl/archives/26
\titleformat{\section}{\Large\scshape\raggedright}{}{0em}{}[\titlerule]
\titlespacing{\section}{0pt}{3pt}{3pt}
%Tweak a bit the top margin
%\addtolength{\voffset}{-1.3cm}

%Italian hyphenation for the word: ''corporations''
\hyphenation{im-pre-se}

%-------------WATERMARK TEST [**not part of a CV**]---------------
\usepackage[absolute]{textpos}

\setlength{\TPHorizModule}{30mm}
\setlength{\TPVertModule}{\TPHorizModule}
\textblockorigin{2mm}{0.65\paperheight}
\setlength{\parindent}{0pt}

%--------------------BEGIN DOCUMENT----------------------
\begin{document}

%WATERMARK TEST [**not part of a CV**]---------------
%\font\wm=''Baskerville:color=787878'' at 8pt
%\font\wmweb=''Baskerville:color=FF1493'' at 8pt
%{\wm 
%	\begin{textblock}{1}(0,0)
%		\rotatebox{-90}{\parbox{500mm}{
%			Typeset by Alessandro Plasmati with \XeTeX\  \today\ for 
%			{\wmweb \href{http://www.aleplasmati.comuv.com}{aleplasmati.comuv.com}}
%		}
%	}
%	\end{textblock}
%}
\pagestyle{empty} % non-numbered pages

\font\fb=''[cmr10]'' %for use with \LaTeX command

%--------------------TITLE-------------
\par{\centering
  {\Huge {\sc Matthew Burke}
  }\bigskip\par}

%--------------------SECTIONS-----------------------------------
%Section: Personal Data
\begin{tabu}{X[1 r]|X[1 c]|X[1 l]}
  +14036049368  & matthew.burke@cantab.net & \url{https://mwpb.uk}
\end{tabu}

%Section: Education
\section{Education}

\begin{tabular}{rl}	
  & \textbf{Macquarie University, Sydney}\\
                     & \textit{Doctor of Philosophy in Mathematics 2011-2015}\\
                     &\\
  & \textbf{Christ's College, University of Cambridge}\\
                     & \textit{Part III Mathematics (MMATH) 2010-2011}\\
                     & \textit{Bachelor's Degree in Mathematics (BA) 2007-2010}\\
                     &\\
\end{tabular}

%Section: Work Experience at the top
\section{Work Experience}

\begin{tabu}{X[1, l]X[1 r]}
{\bf University of Calgary}&\textsc{Sep 2018 - Present}\\
\emph{Sessional instructor}&\textsc{\& Sep 2017 - Dec 2017}
\end{tabu}
\begin{itemize}
  \item Lectured 4 classes of around 230 students for the MATH211 and MATH249 courses.
  \item Used Jupyter notebooks for interactive visualisations and showing intermediate steps.
\end{itemize}

\begin{tabu}{X[1, l]X[1 r]}
{\bf University of Calgary}&\textsc{Sep 2017 - Present}\\
\emph{Postdoctoral scholar}  
\end{tabu}
\begin{itemize}
  \item Initiated a cross-discipline collaboration between the departments of mathematics and computer science with my supervisors.
  Published in a respected peer-reviewed journal.
  \item Provided mentoring support for two PhD students.
  \item Examiner for a PhD student's candidacy examination.
  \item Reviewed papers for two mathematics journals. Organised the University of Calgary Peripatetic Seminar and chaired a session of the Alberta Mathematics Dialogue 2018.
\end{itemize}

\begin{tabu}{X[1, l]X[1 r]}
{\bf MathSpire Ltd.}&\textsc{Nov 2016 - Aug 2017}\\
\emph{Chief technology officer}  
\end{tabu}
\begin{itemize}
  \item Developed a functional reactive cross-platform mobile and desktop application to teach A-level mathematics using interactive graphs, videos and integrated testing.
  \item Created a web front-end and API for teachers to track student progress.
  \item Showcased application at the BETT eduction technology conference.
\end{itemize}

\begin{tabu}{X[1, l]X[1 r]}
{\bf Debate Chamber Ltd.}&\textsc{Jul 2016 - Aug 2016}\\
\emph{Summer school tutor}  
\end{tabu}
\begin{itemize}
  \item Guided classes of around 14 A-level students through university level material using a combination of presentations, exercises and group work.
\end{itemize}

\begin{tabu}{X[1, l]X[1 r]}
{\bf MathSpire Ltd.}&\textsc{Jun 2016 - Nov 2016}\\
\emph{Software developer and content creator}  
\end{tabu}
\begin{itemize}
  \item Created a database of mathematics videos, tests and interactive graphics.
\end{itemize}

\begin{tabu}{X[1, l]X[1 r]}
{\bf Masaryk University, Brno}&\textsc{Oct 2015 - Nov 2015}\\
\emph{Visiting postdoctoral researcher}  
\end{tabu}
\begin{itemize}
	\item Plenary speaker at the multi-disciplinary Eduard Cech Institute Workshop.
	\item Collaborated with researchers in algebra, geometry and theoretical physics.
\end{itemize}

\begin{tabu}{X[1, l]X[1 r]}
{\bf Macquarie University, Sydney}&\textsc{Jan 2013 - Jun 2014}\\
\emph{Tutor}
\end{tabu}
\begin{itemize}
  	\item Demonstrated solutions on the whiteboard for three undergraduate mathematics courses.
\end{itemize}  

\begin{tabu}{X[1, l]X[1 r]}
{\bf Macquarie University Numeracy Centre}&\textsc{Jan 2012 - Dec 2013}\\
\emph{Tutor}
\end{tabu}
\begin{itemize}
  	\item Guided and motivated first year students at the drop-in centre.
\end{itemize}

\begin{tabu}{X[1, l]X[1 r]}
{\bf Blue Tutors}&\textsc{Jun 2010}\\
\emph{GCSE Tutor}
\end{tabu}

\section{Selected Projects}

\begin{itemize}
  \item \textbf{Kaggle Box Office Predictions:} (\url{https://github.com/mwpb/kaggle-projects})\\
  Competition to predict the revenue of films. 
  Uses IPython in Google Colab.
  \item \textbf{Cryptopals Challenges:} (\url{https://github.com/mwpb/matasano-go})\\
  Solutions written in Go. Work in progress: currently first three sets completed.
  \item \textbf{ChessLogBook} (\url{https://mwpb.uk/ChessLogBook})\\
  MacOS app written using Qt to play games on \url{freechess.org} and annotate them offline.
  \item \textbf{Colimits in Coq:} (\url{https://github.com/mwpb/postulated-colimits-in-coq})\\
  Computer verification of a key result in category theory using the Coq proof assistant.
  % \item \textbf{JavaScript Libraries:} 
\end{itemize}

\section{Professional Development and Prizes}
\begin{tabular}{rl}
  2018 & Mitacs online workshop: Managing Project Timelines\\
  2016 & University of Michigan on Coursera: Using Python to Access Web Data\\
  2016 & University of Michigan on Coursera: Using Databases with Python\\
  2008 & Christ's College Whelan Prize for First Class Examination Performance
\end{tabular}

\section{Skills}
\begin{description}
  \item[Over 1 year experience:]F\#, .NET, WebSharper \& Xamarin Frameworks, Git, LaTeX.
  \item[Some experience:] C\#, JavaScript, HTML, CSS, Python, Jupyter Notebook, Coq.
\end{description}

\section{Publications}

\begin{itemize}
  \item Involution algebroids: a generalisation of Lie algebroids for tangent categories, \emph{In preparation}. Available at \url{https://mwpb.uk/static/pdfs/involution-algebroids.pdf}.
    \item A Synthetic Version of Lie's Second Theorem, \emph{Applied Categorical Structures}, 2018.02.06. Available at \url{https://doi.org/10.1007/s10485-018-9518-2}.
    \item Connected Lie Groupoids are Internally Connected and Integral Complete in Synthetic Differential Geometry, \emph{Symmetry, Integrability and Geometry: Methods and Applications}, 2016.06.29. Available at \url{http://www.emis.de/journals/SIGMA/2017/007/}.
\end{itemize}

\section{Selected Conference Presentations}

\begin{itemize}
	\item Canadian Mathematical Society, Summer Meeting, \emph{The Calculus of Infinity Functors and Tangent Categories}, University of New Brunswick, 2018.06.04.
    \item 26th Foundational Methods in Computer Science Workshop, \emph{Tangent Bundles of Groupoids, Pre-groupoids and Torsoids}, Mount Allison University, 2018.06.02.
    \item Seminaire de geometrie et physique mathematique, \emph{Multi-object Lie theory using synthetic differential geometry}, Universite Paris Diderot, Paris 7, 2015.12.04.
    \item Category Theory Seminar, \emph{A Synthetic Version of Lie's Second Theorem}, University of Cambridge, 2015.11.17.
    \item Plenary Speaker at Eduard Cech Institute Workshop, \emph{Synthetic Lie Theory}, Trest, Czech Republic, 2015.10.10.
    \item Category Theory 2014, \emph{Synthetic Lie Theory}, University of Cambridge, 2014.07.04.
\end{itemize}

\end{document}
