\documentclass[10pt]{article}
\usepackage[margin=3cm]{geometry}
\setlength{\parindent}{0in}
\usepackage{tabu}
\usepackage{longtable}
\usepackage{url}
\date{}
\begin{document}

\section*{\Huge Matthew Burke}
\begin{tabu}{@{}X[l6]X[r4]}
Mathematics and Statistics, 612 Campus Place NW & Email: matthew.burke1@ucalgary.ca \\
University of Calgary, 2500 University Drive NW & Telephone: +14036049368\\
Calgary, AB, Canada, T2N 1N4 & Website: \url{https://mwpb.uk}
\end{tabu}

\section*{Academic Employment}
\begin{itemize}
    \item {\bf University of Calgary, Canada}\\
    Postdoctoral Scholar in the Department of Mathematics and Statistics, 2017.09-Present.\\
        Supervisor: Kristine Bauer.
    \item {\bf Masaryk University, Brno, Czech Republic}\\
    Visiting Postdoctoral Researcher in the Department of Mathematics and Statistics, 2015.10-11.
\end{itemize}

\section*{Education}
\begin{itemize}
    \item {\bf Macquarie University, Sydney}\\
    PhD~Mathematics, 2011-2015.\\
    Thesis: Synthetic Lie Theory.\\
    Supervisor: Richard Garner. Co-supervisor: Dominic Verity.\\
    \item {\bf Christ's College, University of Cambridge}\\
    Part III Mathematics (MMath), 2010-2011.\\
    Pass with Merit.\\
    Part III Essay scored 95/100.\\
    \item {\bf Christ's College, University of Cambridge}\\
    Bachelor's Degree in Mathematics (BA), 2007-2010.\\
    Upper Second Class Honours.\\
    Christ's College Whelan Prize in 2008 for First Class Examination Performance.\\Completed seven optional computational projects all of which gained alpha quality marks.
\end{itemize}
\section*{Publications}
\begin{itemize}
    \item A Synthetic Version of Lie's Second Theorem, \emph{Applied Categorical Structures}, 2018.02.06. Available at \url{https://doi.org/10.1007/s10485-018-9518-2}.
    \item Connected Lie Groupoids are Internally Connected and Integral Complete in Synthetic Differential Geometry, \emph{Symmetry, Integrability and Geometry: Methods and Applications}, 2016.06.29. Available at \url{http://www.emis.de/journals/SIGMA/2017/007/}.
\end{itemize}

\section*{Teaching and Work Experience}
\begin{tabu}{X[1]X[5]}
    2017.09-2017.12 & \bf{Sessional Instructor at the University of Calgary}\\
        {} & Lectured Mathematics 249 in the Department of Mathematics and Statistics.\\
    2016.11-2017.08 & \bf{Chief Technology Officer at MathSpire Ltd.}\\
        {} & Extended existing course structure and developed new courses, interactive mathematics lessons and tests for A-level mathematics students. Used F\#, C\#, SQL, JavaScript, ASP.NET and the Xamarin Framework to create cross platform mobile applications, desktop applications and an integrated web API.\\
    2016.07-08 & \bf{Tutor at Debate Chamber Mathematics Summer School}\\
        {} & Guided A-level students through undergraduate level mathematics topics including Linear Algebra, Analysis, Differential Equations, Turing Machines, Infinity and Cryptography.\\
    2016.06-2016.11 & \bf{Software Developer and Content Creator at MathSpire Ltd.}\\
        {} & Created database of mathematics videos, tests and graphics using F\# and SQL.\\
    2013-2014 & \bf{Tutor for Macquarie University}\\
        {} & Demonstrated solutions on the whiteboard for three undergraduate mathematics courses and provided additional individual assistance.\\
    2012 & \bf{Tutor at Macquarie University Numeracy Centre}\\
        {} & Guided and motivated first year students individually and in small groups at the drop-in centre.\\
    2010.06 & \bf{GCSE Tutor for Blue Tutors}\\
\end{tabu}

\section*{Academic Service}
\begin{tabu}{X[1]X[5]}
    2018.01-Present & \bf{Seminar Organiser}\\
{}& Peripatetic Seminar in Category Theory and Logic at the University of Calgary.\\

    2018.05 & \bf{Session Chair}\\
{}& Category Theory and Logic, Alberta Mathematics Dialogue 2018.\\

    2017.12 & \bf{Article Referee}\\
{}& Advances in Mathematics.\\

\end{tabu}

\section*{Professional Development}
\begin{tabu}{X[1]X[5]}
    2018 & \bf{Mitacs Online Workshops}\\
    &{}Managing Project Timelines\\ 

    2016 & \bf{University of Michigan on Coursera}\\
    &{}Using Python to Access Web Data\\ 

    2016 & \bf{University of Michigan on Coursera}\\
    &{}Using Databases with Python\\ 

\end{tabu}

\section*{Talks and Conferences}
\begin{itemize}
    \item Calgary Peripatetic Seminar in Logic and Category Theory, \emph{More Elements of the Theory of Quasi-categories}, University of Calgary, 2018.07.03.
    \item Calgary Peripatetic Seminar in Logic and Category Theory, \emph{Elements of the Theory of Quasi-categories}, University of Calgary, 2018.06.27.
    \item Calgary Peripatetic Seminar in Logic and Category Theory, \emph{A Two Dimensional Setting for the Calculus of Infinity Functors: Part II}, University of Calgary, 2018.05.14.
    \item Calgary Peripatetic Seminar in Logic and Category Theory, \emph{A Two Dimensional Setting for the Calculus of Infinity Functors}, University of Calgary, 2018.04.30.
    \item Calgary Peripatetic Seminar in Logic and Category Theory, \emph{Free co-completion, presheaves and sheaves}, University of Calgary, 2018.03.26.
    \item Calgary Peripatetic Seminar in Logic and Category Theory, \emph{Localisation of Simplicial Presheaf Categories}, University of Calgary, 2018.02.26.
    \item Calgary Peripatetic Seminar in Logic and Category Theory, \emph{The Calculus of Functors using Sheafification}, University of Calgary, 2018.01.22.
    \item Calgary Peripatetic Seminar in Logic and Category Theory, \emph{Sites of Smooth Affine Schemes: Part III}, University of Calgary, 2017.10.13.
    \item Calgary Peripatetic Seminar in Logic and Category Theory, \emph{Sites of Smooth Affine Schemes: Part II}, University of Calgary, 2017.10.06.
    \item Calgary Peripatetic Seminar in Logic and Category Theory, \emph{Sites of Smooth Affine Schemes}, University of Calgary, 2017.09.22.
    \item  Calgary Peripatetic Seminar in Logic and Category Theory, \emph{Lie's Third Theorem using an Intuitionistic Double Negation}, University of Calgary, 2017.09.06.
    \item Category Theory Seminar, \emph{Lie's Third Theorem in Synthetic Differential Geometry},  University of Cambridge, 2017.04.02.
    \item Calgary Mathematics Department Colloquium, \emph{Infinitesimals in Lie Theory}, University of Calgary, 2016.10.07.
    \item Calgary Peripatetic Seminar in Logic and Category Theory, \emph{Lie Theory for Categories using Infinitesimals}, University of Calgary, 2016.09.30.
    \item Seminaire de geometrie et physique mathematique, \emph{Multi-object Lie theory using synthetic differential geometry}, Universite Paris Diderot, Paris 7, 2015.12.04.
    \item Category Theory Seminar, \emph{A Synthetic Version of Lie's Second Theorem}, University of Cambridge, 2015.11.17.
    \item Algebra Seminar, \emph{Lie's Second Theorem}, Masaryk University, Brno, 2015.11.05.
    \item Algebra Seminar, \emph{Jet Part of a Category}, Masaryk University, Brno, 2015.10.28.
    \item Differential Geometry Seminar, \emph{An Introduction to Synthetic Differential Geometry}, Masaryk University, Brno, 2015.10.26.
    \item Plenary Speaker at Eduard Cech Institute Workshop, \emph{Synthetic Lie Theory}, Trest, Czech Republic, 2015.10.10.
    \item Centre of Australian Category Theory, \emph{Jet Categories in the Cahiers Topos (2 talks)}, Macquarie University, 2015.05.13ff.
    \item Category Theory 2014, \emph{Synthetic Lie Theory}, University of Cambridge, 2014.07.04.
    \item MCDC 2014, \emph{A Synthetic Perspective on the Integrability of Lie Algebroids}, Macquarie University, 2014.06.19.
    \item Centre of Australian Category Theory, \emph{A Synthetic Perspective on the Integrability of Lie Algebroids (3 talks)}, Macquarie University, 2014.05.21ff.
    \item MCDC 2013, \emph{Cohomology from the Perspective of Restriction Categories and Atlases}, Macquarie University, 2013.07.04.
    \item MCDC 2012, \emph{Applications of Logic in Differential Geometry}, Macquarie University, 2012.06.15.
    \item Part III talk, \emph{Synthetic Differential Geometry}, University of Cambridge, 2011.
\end{itemize}

\end{document}